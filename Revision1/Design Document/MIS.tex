\documentclass[12pt, titlepage]{article}

\usepackage{fullpage}
\usepackage[round]{natbib}
\usepackage{multirow}
\usepackage{booktabs}
\usepackage{tabularx}
\usepackage{float}
\usepackage{amsfonts}
\usepackage{graphicx}
\usepackage{float}
\usepackage{hyperref}
\hypersetup{
    colorlinks,
    citecolor=black,
    filecolor=black,
    linkcolor=red,
    urlcolor=blue
}
\usepackage[round]{natbib}

\newcounter{acnum}
\newcommand{\actheacnum}{AC\theacnum}
\newcommand{\acref}[1]{AC\ref{#1}}

\newcounter{ucnum}
\newcommand{\uctheucnum}{UC\theucnum}
\newcommand{\uref}[1]{UC\ref{#1}}

\newcounter{mnum}
\newcommand{\mthemnum}{M\themnum}
\newcommand{\mref}[1]{M\ref{#1}}

\title{SE 3XA3: Module Interface Specification \\Poker Project}

\author{Team 12
        \\ Safwan Hossain and hossam18
        \\ Eamon Earl and earle2
        \\ Tyler Magarelli and magarelt
}

\date{April 12, 2022}


\begin{document}

\maketitle

\pagenumbering{arabic}


\section* {Card (ADT)}

Card

\subsection* {Uses}

None

\subsection* {Syntax}

\subsubsection* {Imported Constants}

None

\subsubsection* {Imported Types}

None

\subsubsection* {Exported Access Programs}

\begin{tabular}{| l | l | l | p{5cm} |}
\hline
\textbf{Routine name} & \textbf{In} & \textbf{Out} & \textbf{Exceptions}\\
\hline
greater\verb|_|than & Card & $\mathbb{B}$ &\\
\hline
\end{tabular}

\subsection* {Semantics}

\subsubsection* {State Variables}

$\mathit{suit}: \text{I}$\\
$\mathit{rank}$: $\mathbb{I}$

\subsubsection* {State Invariant}

$1 <= \mathit{suit} <= 4$\\
$2 <= \mathit{rank} <= 14$

\subsubsection* {Assumptions}

None

\subsubsection* {Considerations}

None

\subsubsection* {Access Routine Semantics}

\noindent greater\verb|_|than(C):
\begin{itemize}
\item $rank >= C.rank \Longrightarrow true \phantom{a}|\phantom{a} false$ 

\end{itemize}

\section* {Player (ADT)}

Player

\subsection* {Uses}

Card

\subsection* {Syntax}

\subsubsection* {Imported Constants}

None

\subsubsection* {Imported Types}

Card

\subsubsection* {Exported Access Programs}

\begin{tabular}{| l | l | l | p{5cm} |}
\hline
\textbf{Routine name} & \textbf{In} & \textbf{Out} & \textbf{Exceptions}\\
\hline 
new Player & String, $\mathbb{I}$ & &\\
\hline
clear\verb|_|hand & & &\\
\hline
hasTurn & & $\mathbb{B}$ &\\
\hline 
giveTurn & & &\\
\hline 
takeTurn & & &\\
\hline 
set\verb|_|chips & $\mathbb{I}$ & &\\
\hline 
get\verb|_|chips & & $\mathbb{I}$ &\\
\hline 
take\verb|_|chips & $\mathbb{I}$ & & IllegalArgumentException\\
\hline 
insert & Card & &\\
\hline 
get\verb|_|hand & & Card [ ] &\\
\hline 
\end{tabular}

\subsection* {Semantics}

\subsubsection* {State Variables}

$\mathit{name}: String$\\
$\mathit{chips}: \mathbb{I}$ \\
$\mathit{hand} : \verb|Card [ ]|$\\
$has\verb|_|turn : \mathbb{B}$

\subsubsection* {State Invariant}

$0 <= \mathit{chips}$

\subsubsection* {Assumptions}

None

\subsubsection* {Considerations}

None

\subsubsection* {Access Routine Semantics}

\noindent new Player(s, c):
\begin{itemize}
\item transition : $hand, name, chips, has\verb|_|turn := \epsilon, s, c, False$ 
\end{itemize}

\noindent clear\verb|_|hand():
\begin{itemize}
\item transition : $hand := \epsilon$ 
\end{itemize}

\noindent hasTurn():
\begin{itemize}
\item return : $has\verb|_|turn$ 
\end{itemize}

\noindent giveTurn():
\begin{itemize}
\item transition : $has\verb|_|turn := True$ 
\end{itemize}

\noindent takeTurn():
\begin{itemize}
\item transition : $has\verb|_|turn := False$ 
\end{itemize}

\noindent set\verb|_|chips(c):
\begin{itemize}
\item transition : $chips := c$ 
\end{itemize}

\noindent get\verb|_|chips(c):
\begin{itemize}
\item return : $chips$ 
\end{itemize}

\noindent take\verb|_|chips(c):
\begin{itemize}
\item transition : $chips := chips - c$
\item error : $chips - c < 0 \Longrightarrow IllegalArgumentException$
\end{itemize}

\noindent insert(C):
\begin{itemize}
\item transition : $C \in hand$
\item post-condition: $\forall i \in [0..len(hand) - 2] : hand[i].rank <= hand[i+1].rank$ 
\item description : inserts the card C into hand such that the hand is ordered in ascending order by rank
\end{itemize}

\noindent get\verb|_|hand():
\begin{itemize}
\item return : $hand$ 
\end{itemize}

\section* {Deck (ADT)}

Deck

\subsection* {Uses}

Card
Player

\subsection* {Syntax}

\subsubsection* {Imported Constants}

None

\subsubsection* {Imported Types}

Card
Player

\subsubsection* {Exported Access Programs}

\begin{tabular}{| l | l | l | p{5cm} |}
\hline
\textbf{Routine name} & \textbf{In} & \textbf{Out} & \textbf{Exceptions}\\
\hline
fillDeck & & &\\
\hline
shuffle & & &\\
\hline 
reset & & &\\
\hline 
draw & $\mathbb{I}$& Card [ ] & StackOverflowException\\
\hline 
\end{tabular}

\subsection* {Semantics}

\subsubsection* {State Variables}

$\mathit{deck} : \verb|Card [52]|$\\
$\mathit{flop} : \verb|Card [ ]|$\\
$stack\verb|_|p : \mathbb{I}$

\subsubsection* {State Invariant}

$0 <= stack\verb|_|p <= 51$\\

\subsubsection* {Assumptions}

None

\subsubsection* {Considerations}

Deck is suggested to be implemented as a stack, but the choice is ultimately up to the development team. If it is not implemented as such, the stack\verb|_|p state variable will not be needed and its associated invariant can be disregarded.

\subsubsection* {Access Routine Semantics}

\noindent fill\verb|_|deck():
\begin{itemize}
\item transition : fills the deck stack with all 52 unique playing cards (of type Card) 
\end{itemize}

\noindent shuffle():
\begin{itemize}
\item transition : randomly shuffles the current cards in the deck to a degree wherein the sequence can be expected to be drastically different from the precondition of the deck stack 
\end{itemize}

\noindent reset():
\begin{itemize}
\item transition : returns all 52 unique cards to the deck stack and shuffles the deck, s:= 0
\end{itemize}

\noindent draw(n):
\begin{itemize}
\item transition: removes the top n cards from the deck, and places them into a list 
\item exception: $n >$ remaining cards in the deck $\Longrightarrow StackOverflowException$
\item returns: Card [n] 
\end{itemize}

\section*{Game}

Game

\subsection* {Uses}

Card
Player
Deck

\subsection* {Syntax}

\subsubsection* {Imported Constants}

None

\subsubsection* {Imported Types}

Card
Player 
Deck

\subsubsection* {Exported Access Programs}

\begin{tabular}{| l | l | l | p{5cm} |}
\hline
\textbf{Routine name} & \textbf{In} & \textbf{Out} & \textbf{Exceptions}\\
\hline
new Game & Player [ ], $\mathbb{I}$ & &\\
\hline
startGame & $\mathbb{I}$ & &\\
\hline 
removePlayer & Player & &\\
\hline 
foldPlayer & Player & &\\
\hline 
is\verb|_|round\verb|_|over & & $\mathbb{B}$ &\\
\hline 
dealCards & $\mathbb{I}$ & &\\
\hline
getNextPlayer & & Player & RuntimeException\\ 
\hline 
getCurrentPlayer & & Player & RuntimeException\\
\hline 
giveNextTurn & & &\\
\hline
\end{tabular}

\subsection* {Semantics}

\subsubsection* {State Variables}

\begin{itemize}
    \item deck : Deck
    \item players: Player [ ]
    \item unfoldedPlayers: Player [ ]
    \item currentPlayerIndex: $\mathbb{I}$
    \item nextPlayerIndex: $\mathbb{I}$
    \item minimumCallAmount: $\mathbb{I}$
    \item round\verb|_|over: $\mathbb{B}$
\end{itemize}

\subsubsection* {State Invariant}

\begin{itemize}
    \item there must always be a number of unfolded players less than or equal to the number of players
    \item currentPlayerIndex must always be greater than or equal to 0 and less than the length of unfoldedPlayers
    \item same as above, but for nextPlayerIndex
\end{itemize}

\subsubsection* {Assumptions}

None

\subsubsection* {Considerations}

Implementing players and unfoldedPlayers as dynamic arrays seems ideal.

\subsubsection* {Access Routine Semantics}

\noindent new Game(p\verb|_|list, x):
\begin{itemize}
\item transition:
    \begin{itemize}
        \item deck := new Deck()
        \item players := p\verb|_|list
        \item minimumCallAmount := x
        \item currentPlayerIndex := 0
        \item nextPlayerIndex := 0
        \item round\verb|_|over := False
    \end{itemize}
\end{itemize}

\noindent startGame(c):
\begin{itemize}
\item action: dealcards(c), giveNextTurn()
\end{itemize}

\noindent removePlayer(p):
\begin{itemize}
\item transition: players := \verb|{|players - p\verb|}|\\
\end{itemize}

\noindent foldPlayer(p):
\begin{itemize}
\item transition: unfoldedPlayers := \verb|{|unfoldedPlayers - p\verb|}|\\
\item transition: if no more unfolded players, round\verb|_|over := True
\end{itemize}

\noindent is\verb|_|round\verb|_|over():
\begin{itemize}
    \item return: round\verb|_|over
\end{itemize}

\noindent dealCards(c):
\begin{itemize}
    \item transition: insert c cards into each players hand using Player.insert()
\end{itemize}

\noindent getNextPlayer():
\begin{itemize}
    \item return: the next player to go
    \exception catching out of bounds errors
\end{itemize}

\noindent getCurrentPlayer():
\begin{itemize}
    \item return: the current player to go
    \exception catching out of bounds errors
\end{itemize}

\noindent giveNextTurn():
\begin{itemize}
    \item action: triggers the next players turn
\end{itemize}



\section*{Hand Evaluator}

HandEval

\subsection* {Uses}

Card

\subsection* {Syntax}

\subsubsection* {Imported Constants}

None

\subsubsection* {Imported Types}

Card

\subsubsection* {Exported Access Programs}

\begin{tabular}{| l | l | l | p{5cm} |}
\hline
\textbf{Routine name} & \textbf{In} & \textbf{Out} & \textbf{Exceptions}\\
\hline
evaluate & Card & &\\
\hline
\end{tabular}

\subsection* {Semantics}

\subsubsection* {State Variables}

None

\subsubsection* {State Invariant}

None

\subsubsection* {Assumptions}

This module is made for standard 5 card poker hands, and will be used solely to evaluate such hands

\subsubsection* {Considerations}

It might make sense to use auxiliary functions to evaluate hand states, as different ranks of hands can have similar properties.

\subsubsection* {Access Routine Semantics}

\noindent evaluate(c\verb|_|list):
\begin{itemize}
\item returns : a tuple of integers, the first representing the relative rank of the hand (regarding standard 5 card poker rules), and the second representing the rank of the highest card in the hand for tie breakers 
\end{itemize}

% ============ VIEW ============== %

\section* {GameView}

GameView

\subsection* {Uses}

Card

\subsection* {Syntax}

\subsubsection* {Imported Constants}

None

\subsubsection* {Imported Types}

Card

\subsubsection* {Exported Access Programs}

\begin{tabular}{| l | l | l | p{5cm} |}
\hline
\textbf{Routine name} & \textbf{In} & \textbf{Out} & \textbf{Exceptions}\\
\hline
display & Card & &\\
\hline  
\end{tabular}

\subsection* {Semantics}

\subsubsection* {State Variables}

None

\subsubsection* {State Invariant}

None

\subsubsection* {Assumptions}

None

\subsubsection* {Considerations}

None

\subsubsection* {Access Routine Semantics}

\noindent display(C):
\begin{itemize}
\item behaviour: displays the suit and rank of card c
\end{itemize}

% ============ MAIN CONTROLLER ============ % 

\section* {MainController Module}
    \subsection* {Uses}
        Client, Game, Gameview, MainMenuView, Server
    \subsection* {Syntax}
    
        \subsubsection* {Imported Constants}
            None
        \subsubsection* {Imported Types}
            Client, Game, Server
        \subsubsection* {Exported Access Programs}
        
        \begin{tabular}{| l | l | l | p{5cm} |}
            \hline
            \textbf{Routine name} & \textbf{In} & \textbf{Out} & \textbf{Exceptions}\\
            \hline
            getValidUsername & Scanner & String &\\
            \hline
            getValidOption & Scanner & $\mathbb{Z}$ &\\
            \hline 
            getValidSocketForServer & Scanner & Socket &\\
            \hline 
            hostServer & & & IOException\\
            \hline 
            joinServer & Scanner & & IOException\\
            \hline 
            exitProgram & & &\\
            \hline 
            performMainMenuOperation & Scanner & &\\
            \hline 
            enterProgram & & &\\
            \hline 
        \end{tabular}
        
    \subsection* {Semantics}
    
    \subsubsection* {Environment Variables}
        Keyboard: Scanner(System.in)
        
    \subsubsection* {State Variables}
        $\mathit{username}: String$\\
        $\mathit{socket}: Socket$\\
        $\mathit{client} : Client$\\
    
    \subsubsection* {State Invariant}
        None
    
    \subsubsection* {Assumptions}
        None
    
    \subsubsection* {Considerations}
        None
    
    \subsubsection* {Access Routine Semantics}
    
        \noindent getValidUsername(scanner):
        \begin{itemize}
        \item return : A String that is non-empty if all white spaces are deleted.
        \end{itemize}
        
        \noindent getValidOption(scanner):
        \begin{itemize}
        \item return : $option : \mathbb{Z} | 0 < option <= MAX\_NUM\_OPTIONS$.
        \item description : Returns an integer between 0 and the maximum number of available main menu options.
        \end{itemize}
        
        \noindent getValidSocketForServer(scanner):
        \begin{itemize}
        \item return : A socket that has successfully established a connection with a server.
        \end{itemize}
        
        \noindent hostServer():
        \begin{itemize}
        \item transition : Creates a new server using the user's current IP address.
        \item exception : IO Exception. Can be caused by thousands of issues (IP address, ports, connectivity issues).
        \end{itemize}
        
        \noindent joinServer(scanner):
        \begin{itemize}
        \item transition : Joins an existing server given a server IP address.
        \item exception : IO Exception. Can be caused by thousands of issues (IP address, ports, connectivity issues).
        \end{itemize}

        \noindent exitProgram():
        \begin{itemize}
        \item transition : Shuts down the program.
        \end{itemize}
        
        \noindent performMainMenuOperation(scanner):
        \begin{itemize}
        \item transition : Perform a main menu task, given a number that represents the task to perform. For example, user inputs the number 2 and according to the main menu, number 2 represents the task join a server, so user will join a server.
        \end{itemize}
        
        \noindent enterProgram:
        \begin{itemize}
        \item Transition : Displays welcome screen once. Then displays the main menu and asks the user to select an option in a never-ending loop.
        \end{itemize}
        
% ============ CLIENT CONTROLLER ============ % 
        
\section* {ClientController Module}
    \subsection* {Uses}
        Client, Game, Gameview, PlayerAction, GameInfo
    \subsection* {Syntax}
    
        \subsubsection* {Imported Constants}
            None
        \subsubsection* {Imported Types}
            Client, Game, PlayerAction, GameInfo
        \subsubsection* {Exported Access Programs}
        
        \begin{tabular}{| l | l | l | p{4cm} |}
            \hline
            \textbf{Routine name} & \textbf{In} & \textbf{Out} & \textbf{Exceptions}\\
            \hline
            listenForIncomingMessages &  &  &\\
            \hline
            performGameAction & Scanner &  & IOException\\
            \hline 
            getValidPlayerAction & Scanner & PlayerAction &\\
            \hline 
            getValidBet & Scanner & $\mathbb{Z}$ &\\
            \hline 
            CreateGameInfo & PlayerAction, $\mathbb{Z}$ & GameInfo &\\
            \hline 
        \end{tabular}
        
    \subsection* {Semantics}
    
    \subsubsection* {State Variables}
        $\mathit{player}: Player$\\
        $\mathit{client}: Client$\\
        $\mathit{game} : Game$\\
    
    \subsubsection* {State Invariant}
        None
    
    \subsubsection* {Assumptions}
        None
    
    \subsubsection* {Considerations}
        None
    
    \subsubsection* {Access Routine Semantics}
    
        \noindent listenForIncomingMessages():
        \begin{itemize}
        \item transition : On a separate thread, continuously listens for messages received by $\mathit{client}$ from the server.
        \end{itemize}
        
        \noindent performGameAction(scanner):
        \begin{itemize}
        \item transition : Uses getValidPlayerAction() and getValidBet() to ask the user to make their next move, then stores that information in a new GameInfo object and sends the object to the server from $client$.
        \item exception : Throw IO Exception if there are connectivity issues. Can be caused by thousands of issues (IP address, ports, connectivity issues).
        \end{itemize}
        
        \noindent getValidPlayerAction(scanner):
        \begin{itemize}
        \item return : $playerAction : PlayerAction$
        \item description: Asks the user for a valid player action. If the user input matches a PlayerAction enumerator then return the PlayerAction enmerator. Otherwise ask again.
        \end{itemize}
        
        \noindent getValidBet(scanner):
        \begin{itemize}
        \item return : $amount : \mathbb{Z} | amount >= 0$
        \item description: Asks the user for a valid betting amount. If the user input an integer that is greater or equal to zero then return the integer. Otherwise ask again.
        \end{itemize}
        
        \noindent CreateGameInfo(playerAction, amount):
        \begin{itemize}
        \item return : GameInfo($client.clientID$, $player.name$, playerAction, amount)
        \item description : Creates a GameInfo object with the current player's information (clientID and name) and the move the player wants to make (playerAction and amount).
        \end{itemize}
        
                
% ============ CLIENT ============ % 
        
\section* {Client ADT Module}
    \subsection* {Uses}
    \subsection* {Syntax}
    
        \subsubsection* {Imported Constants}
            None
        \subsubsection* {Imported Types}
            None
        \subsubsection* {Exported Access Programs}
        
        \begin{tabular}{| l | l | l | p{6cm} |}
            \hline
            \textbf{Routine name} & \textbf{In} & \textbf{Out} & \textbf{Exceptions}\\
            \hline
            new Client & Socket, String &  & IOException\\
            \hline
            getClientID &  & String & \\
            \hline 
            setClientID & String &  &\\
            \hline 
            IsConnectedToServer & & $\mathbb{B}$ &\\
            \hline 
            listenForMessage & & Object & IOException, ClassNotFoundException\\
            \hline 
            sendMessage & Object & & IOException\\
            \hline 
            closeEverything &  &  &\\
            \hline
        \end{tabular}
        
    \subsection* {Semantics}
    
    \subsubsection* {State Variables}
        $\mathit{clientID}: String$\\
        $\mathit{playerName}: String$\\
        $\mathit{socket} : Socket$\\
        $\mathit{inputStream}: ObjectInputStream$\\
        $\mathit{outputStream} : ObjectOutputStream$\\

    \subsubsection* {State Invariant}
        None
    
    \subsubsection* {Assumptions}
        None
    
    \subsubsection* {Considerations}
        None
    
    \subsubsection* {Access Routine Semantics}
    
        \noindent new Client(socket, name):
        \begin{itemize}
        \item transition : $self.socket, playerName, inputStream, outputstream := \\ socket, name, new \hspace{3px} ObjectInputStream, new \hspace{3px} ObjectOutputStream$
        \item exception : Throw IO Exception if there are connectivity issues. Can be caused by thousands of issues (IP address, ports, connectivity issues).
        \end{itemize}
        
        \noindent getClientID():
        \begin{itemize}
        \item return : $clientID$
        \end{itemize}
        
        \noindent setClientID(clientID):
        \begin{itemize}
        \item transition : $self.clientID := clientID$
        \end{itemize}
        
        \noindent IsConnectedToServer(scanner):
        \begin{itemize}
        \item return : $socket.IsConnected()$
        \end{itemize}
        
        \noindent listenForMessage():
        \begin{itemize}
        \item return : An Object from $outputStream$ (once recieved).
        \item exception : Throw IO Exception if there are connectivity issues. Can be caused by thousands of issues (IP address, ports, connectivity issues).
        \item exception : Throw ClassNotFoundException if an Object cannot be recieved.
        \end{itemize}
        
        \noindent sendMessage(object):
        \begin{itemize}
        \item transition : Sends in an Object into $inputStream$.
        \item exception : Throw IO Exception if there are connectivity issues. Can be caused by thousands of issues (IP address, ports, connectivity issues).
        \end{itemize}
        
        \noindent closeEverything(playerAction, amount):
        \begin{itemize}
        \item transition : Close $socket, inputStream$ and $outputStream$
        \item description : Closes all sockets,  streams and any connections to the servers.
        \end{itemize}
                
                
% ============ CLIENT HANDLER ============ % 
        
\section* {ClientHandler ADT Module}
    \subsection* {Template Module implements Runnable Interface}
    Client Handler
    \subsection* {Uses}
    Runnable, GameInfo, Game
    \subsection* {Syntax}
    
        \subsubsection* {Imported Constants}
            None
        \subsubsection* {Imported Types}
            GameInfo
        \subsubsection* {Exported Access Programs}
        
        \begin{tabular}{| l | l | l | p{6cm} |}
            \hline
            \textbf{Routine name} & \textbf{In} & \textbf{Out} & \textbf{Exceptions}\\
            \hline
            new ClientHandler & Socket & ClientHandler & \\
            \hline
            run &  &  & \\
            \hline 
            updateClients & GameInfo &  &\\
            \hline 
            closeEverything & & &\\
            \hline
        \end{tabular}
        
    \subsection* {Semantics}
    
    \subsubsection* {State Variables}
        $\mathit{clientUsername}: String$\\
        $\mathit{clientHandlers}:$ static sequence of $ClientHandler$\\
        $\mathit{game}:$ static $Game$\\
        $\mathit{socket}: Socket$\\
        $\mathit{inputStream}: ObjectInputStream$\\
        $\mathit{outputStream} : ObjectOutputStream$\\

    \subsubsection* {State Invariant}
        None
    
    \subsubsection* {Assumptions}
        None
    
    \subsubsection* {Considerations}
        None
    
    \subsubsection* {Access Routine Semantics}
    
        \noindent new ClientHandler(socket):
        \begin{itemize}
        \item return : $self$
        \item transition : $self.socket, inputStream, outputStream, clientHandlers := \\
        socket, new ObjectInputStream(), new ObjectOutputStream, clientHandlers || <self>$
        \item description : initializes $socket$, creates new input and output streams and adds $self$ to $clientHandlers$ (which is a static sequence).
        \end{itemize}
        
        \noindent run():
        \begin{itemize}
        \item transition : Get any commands coming from $outputStream$, input that command into $game$ then send new game information to all clients.
        \item description : Each ClientHandler is responsible for taking in input from a single client connected to a server. Everytime a client sends a command (their game move) to the server, their designated ClientHandler will receive that command and input that command into the game on the server on behalf of the client's name (as if the client had inputted the command directly to the game). Then the ClientHandler will forward the resulting state of the game after the input, synchronizing the game for all clients.
        \end{itemize}
        
        \noindent updateClients(gameInfo):
        \begin{itemize}
        \item transition : For every clientHandler's output stream, write in $gameInfo$ as the output and send.
        \end{itemize}
        
        \noindent closeEverything(scanner):
        \begin{itemize}
        \item transition : Close $socket, inputStream$ and $outputStream$
        \item description : Closes all sockets,  streams and any connections to the servers.
        \end{itemize}
        
        
                        
% ============ SERVER ============ % 
        
\section* {Server ADT Module}
    \subsection* {Uses}
    ClientHandler
    \subsection* {Syntax}
    
        \subsubsection* {Imported Constants}
            None
        \subsubsection* {Imported Types}
            None
        \subsubsection* {Exported Access Programs}
        
        \begin{tabular}{| l | l | l | p{6cm} |}
            \hline
            \textbf{Routine name} & \textbf{In} & \textbf{Out} & \textbf{Exceptions}\\
            \hline
            new Server & ServerSocket &  & \\
            \hline
            startServer &  &  & \\
            \hline 
            closeServer &  &  &\\
            \hline 
        \end{tabular}
        
    \subsection* {Semantics}
    
    \subsubsection* {State Variables}
        $\mathit{serverSocket}: ServerSocket$\\

    \subsubsection* {State Invariant}
        None
    
    \subsubsection* {Assumptions}
        None
    
    \subsubsection* {Considerations}
        None
    
    \subsubsection* {Access Routine Semantics}
    
        \noindent new Server(serverSocket):
        \begin{itemize}
        \item return : $self$
        \item transition : $self.serverSocket := serverSocket$
        \end{itemize}
        
        \noindent startServer():
        \begin{itemize}
        \item transition : Listen for any attempts to connect to $serverSocket$ by a Client. If an attempt is made, try to get the client's socket and create a new ClientHandler (using the client's socket) on a new thread and start the thread.
        \item description : The ClientHandler is responsible for taking in input from a single Client connected to a server. Everytime a Client connects to the server, a new ClientHandler will be created on a new Thread to listen for input from that specific Client. 
        \end{itemize}
        
        \noindent closeServer():
        \begin{itemize}
        \item transition : Close $serverSocket$
        \item description : Closes all sockets,  streams and any connections to the clients.
        \end{itemize}
        
                
                        
% ============ GameInfo ============ % 
        
\section* {GameInfo ADT Module}
    \subsection* {Template Module implements Serializable}
    GameInfo
    \subsection* {Uses}
    PlayerAction
    \subsection* {Syntax}
    
        \subsubsection* {Imported Constants}
            None
        \subsubsection* {Imported Types}
            None
        \subsubsection* {Exported Access Programs}
        
        \begin{tabular}{| l | l | l | p{6cm} |}
            \hline
            \textbf{Routine name} & \textbf{In} & \textbf{Out} & \textbf{Exceptions}\\
            \hline
            new GameInfo & String, String, PlayerAction, $\mathbb{Z}$ & GameInfo & \\
            \hline
            getClientID &  & String & \\
            \hline
            getPlayerName &  & String &\\
            \hline 
            getPlayerAction &  & PlayerID & \\
            \hline 
            getAmount &  & $\mathbb{Z}$ &\\
            \hline 
        \end{tabular}
        
    \subsection* {Semantics}
    
    \subsubsection* {State Variables}
        $\mathit{clientID}: String$\\
        $\mathit{playerName}: String$\\
        $\mathit{playerAction}: PlayerAction$\\
        $\mathit{amount}: \mathbb{Z}$\\

    \subsubsection* {State Invariant}
        None
    
    \subsubsection* {Assumptions}
        None
    
    \subsubsection* {Considerations}
        None
    
    \subsubsection* {Access Routine Semantics}
    
        \noindent new GameInfo(serverSocket):
        \begin{itemize}
        \item return : $self$
        \item transition : $self.clientID, self.playerName, self.playerAction, self.amount := clientID, playerName, playerAction, amount$
        \item description : GameInfo is essentially a data structure that Client and Server will use to communicate.
        \end{itemize}
        
        \noindent getClientID():
        \begin{itemize}
        \item return : $clientID$
        \end{itemize}
        
        \noindent getPlayerAction():
        \begin{itemize}
        \item return : $playerAction$
        \end{itemize}
        
        \noindent getPlayerName():
        \begin{itemize}
        \item return : $playerName$
        \end{itemize}
        
        \noindent getAmount():
        \begin{itemize}
        \item return : $amount$
        \end{itemize}
        
        \noindent toString():
        \begin{itemize}
        \item return : $playerName$ $\|$ " performs the action " $\|$ $playerAction$ $\|$ " for an amount of " $\|$ amount;
        \end{itemize}        
                
                      
% ============ PlayerAction ============ % 
        
\section* {PlayerAction Module}
    \subsection* {Uses}
    PlayerAction
    \subsection* {Syntax}
    
        \subsubsection* {Exported Constants}
            None
        \subsubsection* {Exported Types}
            PlayerAction = \{ \\
            FOLD, \#Player wants to fold \\
            CHECK, \#Player wants to check \\
            CALL, \#Player wants to call \\
            RAISE, \#Player wants to raise \\
            BET \#Player wants to bet \\
            \}

        \subsubsection* {Exported Access Programs}
        
        \begin{tabular}{| l | l | l | p{6cm} |}
            \hline
            \textbf{Routine name} & \textbf{In} & \textbf{Out} & \textbf{Exceptions}\\
            \hline
            isABet & PlayerAction & $\mathbb{B}$ & \\
            \hline
            actionIsValid & String & $\mathbb{B}$ & \\
            \hline
            getActionByString & String & PlayerAction &  IllegalArgumentException\\
            \hline 
        \end{tabular}
        
    \subsection* {Semantics}
    
    \subsubsection* {State Variables}
        None
        
    \subsubsection* {State Invariant}
        None
    
    \subsubsection* {Assumptions}
        None
    
    \subsubsection* {Considerations}
        PlayerAction is an enum class that represent the possible actions a player can make
    
    \subsubsection* {Access Routine Semantics}
    
        \noindent isABet():
        \begin{itemize}
        \item return : $self == BET \vee self == RAISE$
        \end{itemize}
        
        \noindent actionIsValid(action):
        \begin{itemize}
        \item return : True if the String $action$ is a PlayerAction. False if not. 
        \end{itemize}
        
        \noindent getActionByString(action):
        \begin{itemize}
        \item return : Corresponding PlayerAction that matches $action$
        \item exception : Throw IllegalArgumentException if there is no  string value for PlayerAction that matches $action$
        \end{itemize}
         
        
        
% ============ MainMenuView ============ % 
        
\section* {MainMenuView Module}
    \subsection* {Uses}
    \subsection* {Syntax}
    
        \subsubsection* {Imported Constants}
            None
        \subsubsection* {Imported Types}
            None
        \subsubsection* {Exported Access Programs}
        
        \begin{tabular}{| l | l | l | p{6cm} |}
            \hline
            \textbf{Routine name} & \textbf{In} & \textbf{Out} & \textbf{Exceptions}\\
            \hline
            displayWelcomeScreen &  &  & \\
            \hline
            displayMainMenu &  &  &\\
            \hline 
            askForMenuOption &  &  & \\
            \hline 
            displayInvalidMenuOption &  &  &\\
            \hline 
            askForUsername &  &  & \\
            \hline
            displayInvalidUsername &  &  &\\
            \hline 
            displayServerIPAddress & String &  & \\
            \hline 
            displayServerJoinMenu &  &  &\\
            \hline 
            displayFailedToConnectToServer & String &  & \\
            \hline
            displaySuccessfullyStartedServer &  &  &\\
            \hline 
            displaySuccessfulConnection &  &  & \\
            \hline 
            displayWaitingForHost &  &  &\\
            \hline 
            displayExitingProgram &  &  &\\
            \hline 
        \end{tabular}
        
    \subsection* {Semantics}
    
    \subsubsection* {State Variables}
        None
        
    \subsubsection* {State Invariant}
        None
    
    \subsubsection* {Assumptions}
        None
    
    \subsubsection* {Considerations}
        None
    
    \subsubsection* {Access Routine Semantics}
        
        \noindent displayWelcomeScreen():
        \begin{itemize}
        \item output : out := Display a welcome screen message.
        \end{itemize}
        
        \noindent displayMainMenu():
        \begin{itemize}
        \item output : out := Display a main menu options.
        \end{itemize}
        
        \noindent askForMenuOption():
        \begin{itemize}
        \item output : out := Display a message asking the user to enter in their desired main menu option.
        \end{itemize}
        
        \noindent displayInvalidMenuOption():
        \begin{itemize}
        \item output : out := Display a message saying that the main menu option they entered is invalid.
        \end{itemize}
        
        \noindent askForUsername():
        \begin{itemize}
        \item output : out := Display a message asking the user to enter in their desired username.
        \end{itemize}
        
        \noindent displayInvalidUsername():
        \begin{itemize}
        \item output : out := Display a message saying that the username they entered is invalid.
        \end{itemize}
        
        \noindent displayServerIPAddress(serverIP):
        \begin{itemize}
        \item output : out := Display a message saying that the IP address of the server is $serverIP$.
        \end{itemize}
        
        \noindent displayServerJoinMenu():
        \begin{itemize}
        \item output : out := Display a join to server menu.
        \end{itemize}
        
        \noindent displayFailedToConnectToServer():
        \begin{itemize}
        \item output : out := Display an error message saying the user failed to connect to the server.
        \end{itemize}
        
        \noindent displaySuccessfullyStartedServer():
        \begin{itemize}
        \item output : out := Display a success message saying the user successfully started the server.
        \end{itemize}
        
        \noindent displaySuccessfulConnection():
        \begin{itemize}
        \item output : out := Display a success message saying the user successfully connected to the server.
        \end{itemize}
        
        \noindent displayWaitingForHost():
        \begin{itemize}
        \item output : out := Display a message saying that the server is waiting for the host to start the game.
        \end{itemize}
        
        \noindent displayExitingProgram():
        \begin{itemize}
        \item output : out := Display an exit game message.
        \end{itemize}
        
\bibliographystyle {plainnat}
\bibliography {MG}

\end{document}

