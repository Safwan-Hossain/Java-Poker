\documentclass{article}
\usepackage[utf8]{inputenc}
\usepackage[margin=1.3in]{geometry}
\usepackage{fancyhdr}
\usepackage{tabularx}
\usepackage{booktabs}
\pagestyle{fancy}

\title{Problem Statement \\ Poker Project}
\author{
  Safwan Hossain\\
  \texttt{hossam18}\\
  \and
  Tyler Magarelli\\
  \texttt{magarelt}\\
  \and
  Eamon Earl\\
  \texttt{earle2}
}
\date{April 12, 2022}

\begin{document}
\maketitle
\newpage
% \lhead{Safwan Hossain, hossam18 \\
%        Tyler Magarelli, magarelt  \\ 
%        Eamon Earl, earle2}
% \setlength{\headheight}{32.09pt}

\begin{table}[bp]
\caption{\bf Revision History}
\begin{tabularx}{\textwidth}{p{3cm}p{2cm}X}
\toprule {\bf Date} & {\bf Version} & {\bf Notes}\\
\midrule
January 18, 2022 & 1.0 & Initial Draft\\
April 12, 2022 & 2.0 & Revision 1\\
\bottomrule
\end{tabularx}
\end{table}

\section*{Problem Statement}
\subsection*{Introduction}
    Poker is a world renowned card main.model.game in which players must use their understanding of probability, wit and deceptive strategy to best their opponents and win money. While poker is very commonly played face-to-face on a table, our aim as developers is to make any variation of poker easily accessible from anywhere. The current software we begin with sets the framework for randomizing poker hands and analyzing them and by using this we will be able to develop a complete poker experience.
\subsection*{Importance}
    Poker demands the use of specific equipment in order to play the main.model.game correctly. A main.model.game of poker requires a deck of cards, poker chips and a sufficient number of players present, making it a difficult main.model.game to play on demand. Our solution would not only minimize the necessary materials needed to play, but would also provide the user freedom in terms of where and when to play, as well as the ability to play locally with friends or alone in single-player variations.

\subsection*{Context}
    The stakeholders for this project are the developers, publishers, the players, as well as the professor and TAs of this course. Running our solution in a Java environment provides the software developers the correct environment to be able to easily adapt and tailor the software towards a simple yet effective interface while providing the user an easy way to play whenever desired with very little initial setup. Our software will be able to run on nearly any machine that is capable of running JVM on their system and any operating system such as MacOSX, Windows or Linux.

\end{document}

